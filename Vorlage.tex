% Externes Stylsheet
\documentclass{htwchur}



% für abbildungsverzeichnis
\makeatletter
\renewcommand\@makefntext[1]{\rightskip=2.5em\leftskip=3em\hskip-2em\@makefnmark#1}
\makeatother

\title{Latex Vorlage}
\author{M.Muster}


\makeindex
\makeglossary


\begin{document}

\newcommand{\titel}{Latex Vorlage}
\newcommand{\untertitel}{Inoffizielle Vorlage}

\newcommand{\authors}{M.Muster}
\newcommand{\lecturer}{xyz}
\newcommand{\module}{Vorlage}
\newcommand{\program}{Bsc Informationswissenschaft}
\newcommand{\major}{Data and Information Management}
\newcommand{\semester}{Herbstsemester xyz}
 

\begin{titlepage}



\vfill

  \begin{center}
     \textbf{\textsf{\Large \titel}}
  \end{center}
  \begin{center}
    \textsf{\large \untertitel}
  \end{center}


\vfill

  \begin{flushleft}
    \begin{tabular}{l l}
      \textsf{Autoren:}			& \authors \\
      \textsf{Betreuer:}		& \lecturer \\
      \textsf{Modul:}				& \module \\
      \textsf{Studiengang:}	& \program \\
      \textsf{Vertiefung:}	& \major \\
      \textsf{Semester:}		& \semester
    \end{tabular}
  \end{flushleft}

\vfill

\end{titlepage}


\begin{abstract}
Diese Arbeit entstand im Zuge des Modules xyz
im Studiengang Information Science an der Hochschule für Technik und Wirtschaft (HTW) Chur. Sie behandelt xyz 
\\\\
\textbf{Schlagwörter: }Vorlage, Wissenschaftlichkeit

\end{abstract}



\newpage
\pagenumbering{Roman}
\setcounter{page}{2}
%%%% Inhaltsverzeichnis
\tableofcontents

\newpage
%%%% Abkürzungserzeichniss
\section*{Abkürzungsverzeichnis}

\begin{acronym}[Bash]
	\acro{HTW}{Hochschule für Technik und Wirtschaft}
\end{acronym}

\newpage

%%%% Abbildungsverzeichnis
\section*{Abbildungsverzeichniss}
\setlength{\cftfignumwidth}{3cm}
\setlength{\cftfigindent}{0cm}
\renewcommand{\cftfigpresnum}{Abbildung }
\renewcommand{\cftfigaftersnum}{:}
\makeatletter
\@starttoc{lof}% Print List of Figures
\makeatother


%%%% Tabellenverzeichnis
\section*{Tabellenverzeichnis}
\phantomsection
\setlength{\cfttabindent}{0cm}
\setlength{\cfttabnumwidth}{3cm}
\renewcommand{\cfttabpresnum}{Tabelle }
\renewcommand{\cfttabaftersnum}{:}
\makeatletter
\@starttoc{lot}% Print List of Tables
\makeatother

%%%% Start der Nummerierung
\newpage
\pagenumbering{arabic}
\setcounter{page}{1}

\section{Ziel und Nutzen} \label{sec:ziel_und_nutzen}
\subsection{Ausgangslage} \label{sec:ausgangslage}
\lipsum[1-1]


\section{Arten} \label{sec:zielsetzung_und_forschungsfrage}

\subsection{Liste}

\begin{itemize}
	\setlength{\parskip}{0pt}
	\item Dies ist ein Listeneintrag
	\item Das auch
\end{itemize}

\subsection{Tabelle}

\begin{table}[H]
\renewcommand{\arraystretch}{1.5}
\centering
\begin{tabular}{|l|l|}
	\hline
	\textbf{Python}       & \textbf{Java}                   \\ \hline
	Schön & Hässlich                       \\ \hline
	Leicht zu lernen         & java??                                 \\ \hline

\end{tabular} \caption{Dies ist eine Beispielstabelle}	
\end{table}

\subsection{Grafik}

\begin{figure}[H]
	\centering
	\includegraphics[scale = 0.85]{Bilder/HTW_Logo.png}
	\caption[Logo der HTW Chur]{Logo der HTW Chur}
	\label{fig:workflow}
\end{figure}


\subsection{Seitenverweise} \label{sec:Verweis}
In Kapitel \textit{\ref{sec:ausgangslage} \nameref{sec:ausgangslage}}\footnote{Seite \pageref{sec:ausgangslage}.} xyz

\subsection{Abkürzungen}
\ac{HTW}

\subsection{Zitieren} \label{sec:Zitieren}


\subsubsection{zitation}
hallo welt\cite{IPTC}

\subsubsection{Direkt Mittig}
" \textit{Ohne Top-Management-Unterstützung stehen die Wissensmanager [...] auf verlorenem Posten.}" \cite{IPTC}

\subsubsection{Zitat hervorheben}
\begin{displayquote}
	\textit{"\lipsum[1-1]"} \cite{IPTC}
\end{displayquote}


\newpage

\bibliographystyle{apacite}
\bibliography{Literatur_vorlage}



% Dokumentenanhang
\newpage

\section{Anhang}
\newcounter{at_counter}
\newcommand{\atcount}[1]{\refstepcounter{at_counter}}

\includepdf[pages=-]{Anhang/Vorlage_HTW.pdf}



\end{document}
